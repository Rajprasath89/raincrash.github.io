\documentclass[margin]{res}  
\textwidth=5.2in 
\begin{document} 

\address{\Large {\bf{Sricharan Chiruvolu}}\\[10pt]{\bf (+91) 9035 886166 } \\ {\bf \underline{sricharanized@gmail.com}}
        }
\address{\# 430, Mathura block \\ Amrita School of Engineering  \\
         Kasavanahalli, Bengaluru \\ India - 560035}
 
\begin{resume}

	
\section{Interests} 
 Website design and development, computer graphics, free and open source software

\section{Academic Qualifications}
{\bf B.Tech, 6th Semester} \hfill 2012-2016
 \\Computer Science and Engineering\\ Amrita School of Engineering, Bengaluru \hfill CGPA(current) {\bf 8.80/ 10.00}

{\bf 12th Class} \hfill 2012
\\NRI Junior College, Hyderabad\hfill Percentage {\bf 91.6\% }

{\bf 10th Class}\hfill 2010
 \\ Kendriya Vidyalaya, Hyderabad\hfill CGPA {\bf 9.60/ 10.00}


\section{Experience}
	
	{\bf Google Summer of Code - 2015, PySoy} \hfill Summer  2015
	\\Google Inc.
	\\Currently working as a student developer for Google.\\ Will be developing open source software with Google Inc.'s sponsorship for PySoy (Copyleft Games), a 3D cloud game engine.  More information on my project can be found at: \underline{http://goo.gl/frUCNa}.

 {\bf Introduction to Linux}- LFS101x.2 \hfill Winter  2014
	\\LinuxFoundationX
	\\Successfully completed a certification course with an Honor Code certificate, offered by the Linux Foundation.
	
	{\bf J2SE professional certification} \hfill Summer  2014 
	\\Aptech India Private Ltd.
	\\Successfully completed a certification course on Java programming and application development.

 {\bf In-plant training} \hfill Winter 2013 
 \\HCL Technologies
 \\Gained industrial exposure of software development. Attended workshops on Android application development, PC hardware e.t.c.

\section{Technical Expertise}
\begin{itemize}
	\item Programming languages - C++, Python, Genie and Java
	\item Website design and development%  - Python (Django and Flask), Javascript(MEAN) stack
	\item 3D Graphics Programming - OpenGL and GLSL %and Game Development(Unity)
	\item IDE’s: QtCreator, CodeBlocks, MATLAB, Android Studio, Eclipse
	\item Knowledge on build tools (Cmake, GNU make) and version control systems (git, mercurial)
	\item Designing knowledge: Adobe Photoshop and Illustrator
	\item Documentation: Doxygen (generation) and LaTeX (preparation)
\end{itemize}


\section{Open-source Contributions}
{\bf Copyleftgames} \\ Currently contributing in the development of Pysoy as a student developer from Google, under {\bf Google summer of code} program. \\ Working on enhancement in the lighting and shadow rendering. Worked on reproducing and fixing bugs, enhancing the physics engine and game render cycles. \\Details: \underline{http://copyleftgames.org/} and \underline{http://goo.gl/frUCNa}

{\bf Codrspace} \\ Worked on minor enhancements of the python backend.\\ Currently working as a UI/UX developer.  \\Details: \underline{http://codrspace.com/}


 \section{Projects done}
{\bf RayBox} \hfill Jan 2015 \\ RayBox is an image synthesis system based on ray-tracing which includes reflections, refractions, soft shadows and depth of field.

{\bf ClubBox} \hfill Nov 2014 \\ Developed a python based club management system where club in 
charges and faculty can interact with the club executives and organized events effectively.

{\bf ScreamBox} \hfill May 2014 \\ Developed a blogging website using Django Framework where users can blog, follow others send personal messages and interact.\\ Currently hosted at \underline{http://sricharan.herokuapp.com/}.

{\bf Few Personal Projects} \\ Developed various OpenGL based models, a maze-solver using image processing techniques, a simple loadable kernel module, a google app engine based blogging site, a few bash scripts, scripts using Twitter, Facebook graph, openweather and various other RESTful APIs e.t.c. \\Details: \underline{http://github.com/sricharanized/}.

 \section{Ongoing Projects}
{\bf ClassBox} \\ A complete productivity suite for schools and colleges to interact effectively. Features include an interactive drawing board, chat rooms, dedicated forums, quiz generation and analytics, push notifications e.t.c. This application is being developed using MEAN javascript stack. A companion android application is also under development.



\section{Volunteer  Activities} 
{\bf Executive member/Office bearer}\hfill  2014-2015\\ Forum for Aspiring Computer Engineers [FACE] \\Organized events and workshops, developed websites and designed posters for college fests and events. Conducted seminars and workshops that are aimed at imparting knowledge in the field of computer science at Amrita School of Engineering, Bengaluru. \\ Details: \underline{https://faceaseb.wordpress.com}
                

\section{Hobbies}
\\Enjoying sitcoms, reading manga (Japanese comics), watching anime.

\section{Personal Details}
\begin{tabular}{l p{3.5in}}
      \bf{Date of Birth} & \hfill 12th October 1994\\
			\bf{Linguistic Proficiency} & \hfill  English, Hindi and Telugu\\
 \end{tabular}
\section{Links}
\begin{tabular}{l p{3.5in}}
      \bf{Technical Blog} & \hfill \underline{http://sricharanized.wordpress.com/}\\
			\bf{Github} & \hfill  \underline{http://github.com/sricharanized/}\\
 \end{tabular}


\end{resume}
\end{document} 

